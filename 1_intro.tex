高レベル放射性廃棄物(HLW)の地層処分において,ベントナイト緩衝材には,地下水と放射性核種の移行を抑制する
役割を長期間に渡って果たすことが求められる.
ベントナイト粘土の主成分であるモンモリロナイトは,ナノ・メートルスケールの微細な間隙をもつ多孔質構造を作り,
間隙構造や水分分布と移動によって熱や物質輸送特性が変化する.例えば,積層した粘土層間では粘土表面に間隙水が
強く水和されていて水分や物質の移動が起こりにくい.一方,粘土層外に残されたより大きな間隙に保持された水分は,
層間水に比べて移動しやすいと考えられる.このような,水分移動やそれに伴う物質輸送特性が,粘土緩衝材のスケール
でどのような透水や拡散挙動となって現れるかは,間隙率や飽和度だけでなく,間隙の幾何構造や間隙径の分布
に影響される.従って,粘土含水系における水分分布や,間隙の幾何形状とスケールについて理解することは,
マクロな透水や物質輸送特性を正確に評価する上で重要である.

%粘土含水系の間隙や間隙水の状態を顕微鏡下で観察することは難しく,ナノからマイクロメートルスケールでの多孔質構造に関する正確な理解はあまり進んでいない.
粘土鉱物の結晶構造や水和および膨潤挙動については,分子動力学(Molecular Dynamics: MD)法による水和や膨潤についてのシミューレション解析が精力的に行われてきた.
MD法では,物質を構成する原子や分子間の相互作用則を与える他に系の性質を仮定することなく,粘性や拡散係数をはじめとするクロ物性を評価することができる.
ただし,構成原子全ての自由度を考慮する全原子MDは計算負荷が非常に高く,MDモデルの空間スケールは概ねナノメートル程度に限定される.
従って,多数の粘土分子から構成される系を全原子MD計算で解析することは現状では困難である.
これに対して著者らは,ナノからマイクロメートルスケールにおける粘土含水系の挙動を調べることを目的として,粗視化分子動力学の方法を用いた
数値シミュレーション手法の開発に取り組んで来た.このような空間スケールは,連続体近似が可能なmm程度のスケールと,分子シミュレーションが
必要となるナノスケールの中間に当たることから,粗視化MDの方法をメソスケールMD(メソMD)と称している.メソMDでは,物質を構成する原子のうち,
概ね剛体的に挙動することが予想される原子のグループを一つの仮想的な粒子(粗視化粒子)で表現することで,計算モデルの自由度を削減する.
これにより,より大きな時空間スケールで物質をモデル化し,その挙動を計算機シミュレーションで調べることを可能にする.

著者らが開発を行っている粘土含水系のメソMDシミュレーションでは,モンモリロナイト粘土分子の単位構造とそこに水和された間隙水を一つの粗視化粒子で表し,
粗視化粒子間の相互作用力は全原子MD計算の結果から与えている.この方法により,指定された温度,圧力で圧縮したときに生じる粘土含水系の凝集挙動を調べている.
また,メソMD解析で得られた粘土含水系の組織構造モデルを用いて拡散や変形解析を行い,微視的多孔質構造を考慮したマクロ拡散係数や弾性係数の評価が可能
であることを示してきた.また,粘土分子が互いに接近しているときには,近接する粗視化粒子間で水分移動を許容することで,不自然な気泡等を残すことなく,
より安定した組織構造が得られることも,昨年度までの研究によって明らかにしている.一方で,メソMD解析の結果として得られる組織構造が,
あるシミュレーション条件の変更に対して有意に変化するかどうかを判定し,組織構造形成に与える影響の程度を定量的に評価する手法の開発は
これまで行っていない.そのような手法の開発は,シミュレーション結果の妥当性の検証や,マクロ物性に強い影響を与える因子の特定と定量化,
ひいては,マクロ物性の発現メカニズムを理解する上で取り組むべき課題と言える.
%
メソMDシミュレーションの検証を経て,不飽和粘土のマクロ物性に関する定量的予測技術を確立できれば,
例えば,緩衝材再冠水時の過渡的挙動を有限要素法等の手法で調べる際に,変形や熱,水および物質輸送
に関するパラメータを種々の温度や圧力,水分状態において実験を行うことなく与えることも将来的には
実現することが期待できる.
%

以上を踏まえ本年度の研究では,メソMD計算で得られた組織構造を解析し,定量的に特徴づける方法の開発と実装を行った.
具体的には,メソMDシミュレーション結果からX線回折パターンと動径分布関数を合成することで,粘土分子の平均的な積層数や
層間距離を評価することを可能とする.これらは実測可能かつ,物理的な意味が明確な量であるため,粘土含水系の積層構造形成
メカニズムの理解や,その影響因子の抽出と序列化,計算モデルの妥当性の検証に有用となる.また,シミューレションで得られた
組織構造の配向性と局所密度分布の評価方法と結果を示す.これらの量は,今後,組織構造と物質輸送特性の間に存在する
因果関係を調べる上で必要な技術になるものと考えられる.

以下では,上記の組織構造解析を行うために用意した2種類の粘土含水系モデルを示す.
次に,フーリエ変換を利用したX線回折パターンの合成方法と,粘土分子の積層数を評価するために考案した
動径分布関数について述べる.続いて,組織構造が有する配向性の評価方法と,メソスケール間隙を抽出するための
局所密度の定義と計算方法を示す.最後に,粘土含水系の組織構造を特徴付けるこれら一連の量をメソMD計算結果から
実際に評価し,その結果から読み取ることができる情報について検討する.その際,2種類メソMDモデルで
得られる組織構造について特徴量を互いに比較し,メソMD計算条件の違いによって組織構造とその特徴量に有意な
差が現れることを示す.

