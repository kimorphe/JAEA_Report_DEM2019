高レベル放射性廃棄物(HLW)の地層処分において、ベントナイト緩衝材は、地下水と放射性各種の移行を抑制する役割を長期に渡って果たすことが求められる。ベントナイト粘土の主成分であるモンモリロナイトは、ナノスケールの微細な間隙をもつ多孔質構造を作り、熱や物質輸送特性は間隙の構造や水分の分布や移動によっても変化する。例えば、積層した粘土層間では粘土表面に間隙水が強く水和されており、水分移動が起こりにくく、積層した粘土間に残されたより大きな間隙中にあるバルク水は比較的移動しやすいと考えられる。これらが粘土緩衝材のスケールでどのような透水性となって現れるかは、間隙率や飽和度だけでなく、間隙の幾何構造や間隙径の分布に影響されることは明らかで、従って、粘土含水系における水分の分布や間隙の幾何学的形状やスケールについて理解することは、透水性や物質輸送特性の正確な評価を行う上で重要である。しかしながら、粘土含水系のナノスケールでの粘土鉱物や間隙、間隙水の状態を顕微鏡下で直接観察することは難しく、これら微視的な多孔質構造に関する正確な理解はあまり進んでいない。

粘土鉱物の結晶構造や水和および膨潤挙動については、分子動力(MD)学法に基づくシミューレション解析が精力的に行われてきた。MD法では、物質を構成する原子や分子間の相互作用則を与える他、系の性質を仮定することなく粘性や拡散係数、誘電率といったマクロ物性を評価することができる。しかしながら、構成原子全ての自由度を考慮する全原子MDは計算負荷が非常に高く、取り扱うことのできるモデルの時空間的スケールは概ねナノメートル、ナノ秒の範囲である。従って、多数の粘土分子から構成される系を全原子MD計算で解析することは、現状では困難である。これに対して著者らは、粗視化分子動力学の方法を用い、ナノからマイクロメートルスケールでの、粘土含水系の挙動を調べることを目的とした数値シミュレーション技術の開発に取り組んで来た。このような空間スケールは、連続体近似が可能なmm程度のスケールと、分子シミュレーションが必要となるナノスケールの中間に当たることから、粗視化MDの方法をメソスケールMD(メソMD)と称している。メソMDでは、物質を構成する原子のうち、相対位置を大きく変化させることなく運動すると予想される原子のグループを、一つの仮想的な粒子(粗視化粒子)で表現することで、モデルの自由度を削減する。これにより、より大きなスケールで物質をモデル化しその挙動を計算機シミュレーションによって調べることを可能にする。

著者らが開発を行っている粘土含水系のメソMDシミュレーションでは、粘土結晶の単位構造と、そこに水和された間隙水を一つの粗視化粒子で表し、粗視化粒子間の相互作用力は全原子MD計算の結果から与えている。この方法により、指定された温度、圧力で圧縮したときに生じる粘土含水系の凝集挙動を調べている。また、メソMD解析で得られた粘土含水系の組織構造モデルを用いて拡散や変形解析を行い、微視的な多孔質構造を考慮したマクロ拡散係数や弾性係数の評価が可能であることを示してきた。また、粘土分子同士が接近しているときには、近接する粗視化粒子間で水分の移動を許容することで、不自然な気泡を残すことなく、安定した組織構造が得られることも昨年度までの研究によって明らかとすることができた。一方で、メソMD解析の結果として得られた組織構造が、特定のシミュレーション条件において有意に変化したかを判定し、組織構造の形成に与える重要な因子を特定することは未だ実現できていない。また、メソMDメソ計算により、拡散係数などのマクロ物性が組織構造のどのような特徴を反映して決まるかといった点を理解する方法も明らかとなっていない。
これらの問題は、シミュレーション結果の妥当性の検証や、マクロ物性に強い影響を与える因子の特定と定量化、ひいては、マクロ物性の発現メカニズムを理解する上で解決すべく残された課題となっている。

以上を踏まえ、本年度は、メソMD計算で得られた組織構造を解析し、定量的に特徴づける方法の開発と実装を目的として研究を行った。具体的には、メソMDシミュレーションの結果から、X線回折パターンや動径分布関数を合成することで、粘土分子の平均的な積総数や層間距離を評価することを可能とする。これらは、実測可能かつ、物理的な意味が明確なものであるため、粘土含水系の積層構造形成メカニズムの理解や影響因子の抽出と序列、計算モデルの妥当性の検証が可能となる。また、シミューレションで得られた組織構造の局所密度を評価ならびに可視化することで、層間と層外の間隙を区別する方法を提案する。これは、今後、物質輸送に与える組織構造の影響を調べる際に必要となる技術と考えられる。
%これにより、メソMD計算の結果が、計算条件によって有意に変化下かどうかを判定することや、計算結果として得られる組織構造の成因を理解するために有用な情報を得ること、

以下では、上記の組織構造解析を行うために用意した、2種類の粘土含水系のモデルを次節に置いて示す。続いて、フーリエ変換を利用したX線回折パターンの合成方法、粘土の積層数を評価するために考案した動径分布関数と、メソスケール間隙を抽出するための局所密度の定義時計方法を順に述べる。これら一連のプロットを、メソMD計算結果から実際に計算して示し、そこから読み取ることができる情報について述べる。最後に、2種類メソMDモデル対して得られる組織構造の特徴量を比較することで、メソMDの計算条件が組織構造のどのような特徴の差として現れるかを議論する。

