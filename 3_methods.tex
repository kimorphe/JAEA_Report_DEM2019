メソMD解析の結果として得られた組織構造を,定量的に特徴づけるために次の4つの分布を求める。
\begin{itemize}
	\item X線回折(X-ray diffraction:XRD)パターン
	\item 動径分布関数(Radial distribution function:RDF)
	\item 局所密度分布
	\item 粗視化粒子方向の確率密度分布
\end{itemize}
以下、これらの分布関数の定義と求め方を示すとともに,モデル1について実際に計算した
4種類の分布関数とそこから読み取ることのできる情報について述べる.
\subsection{X線回折(XRD)パターン}
試料によって散乱されたX線の観測方向による散乱X線強度の変化をプロットしたグラフを
X線回折パターンと呼ぶ.X線源がmonochromaticな平面波を試料に照射していると仮定するとき、
位置$\fat{x}$における入射X線$u(\fat{x})$は波数ベクトル$\fat{k}^{in}$を用いて
\begin{equation}
	u^{in}(\fat{x})=I_0 \exp(i\fat{k}^{in}\cdot\fat{x})
	\label{eqn:uin}
\end{equation}
と表すことができる。今、試料を構成する物質の作る電荷密度分布を$\rho(\fat{x})$、
散乱X線の波数ベクトルを$\fat{k}_{sc}$とすれば、無限遠方での散乱X線を
\begin{equation}
	u^{sc}(2\theta)=I_0 \int \rho(\fat{x})\exp(i\fat{k}\cdot\fat{x})d\fat{x}, \ \ 
	\left(\fat{k}=\fat{k}^{sc}-\fat{k}^{in}\right)
	\label{eqn:usc}
\end{equation}
と表すことができる。
ただし$2\theta$は$\fat{k}^{in}$と$\fat{k}^{sc}$が成す角を表す.
X線計測で実際に測定することができるのはX線強度だけである.
散乱X線の強度は式(\ref{eqn:usc})の絶対値をとり$\left| u^{sc}(\fat{x})\right|$で,
与えられ,
\begin{equation}
	\left|u(2\theta)\right|
	=
	\left|\int \rho(\fat{x}) \exp(i\fat{k}\cdot \fat{x}) d\fat{x} \right|
	\label{eqn:I2th_bar}
\end{equation}
と書くことができる。この量は,入射X線の伝播方向を$\varphi$にも依存するので,
\begin{equation}
	I_\varphi(2\theta):
		=\frac{\left|u(2\theta)\right|}{I_0}
		=
	\left|\int \rho(\fat{x}) \exp(i\fat{k}\cdot \fat{x}) d\fat{x} \right|
	\label{eqn:Iphi}
\end{equation}
と表すことにする.実際には試料は様々な方向を向いていると考えれば、試料からみて
X線はあらゆる方向から入射されることになる。その時観測される散乱X線強度$I(2\theta)$は
式(\ref{eqn:Iphi})を$\varphi$について積分した
\begin{equation}
	I(2\theta)=\int I_{\varphi}(2\theta) d\varphi
	\label{eqn:XRD}
\end{equation}
で与えられることが分かる.式(\ref{eqn:XRD})を角度$2\theta$の関数として表したものが
X線回折パターンであると考えることができる。
従って,メソMDで得られた組織構造から電荷分布を決定し,式(\ref{eqn:Iphi})と式(\ref{eqn:XRD})
を求めれば、シミュレーション結果に対応したXRDパターンを合成することができる。
このとき,式(\ref{eqn:Iphi})の積分は電荷密度分布のフーリエ変換を計算することによって評価する
ことができる。
本研究では、電荷密度$\rho(\fat{x})$の最も簡単なモデルとして,
固相領域を表す次のような特性関数$\Gamma(\fat{x})$を用いる
\begin{equation}
	\Gamma(\fat{x})=\left\{
		\begin{array}{cc}
			1 &  \fat{x} \in D_s \\
			0 &  otherwise
		\end{array}
	\right.
	\label{eqn:Gamma}
\end{equation}
ただし$D_s$は組織構造モデルにおいて粘土分子が占める固相領域を意味する.
このモデルでは,電荷は粘土分子内で均一に分布しており、粘土分子内の電荷分布による
散乱の指向性や、水和水によるX線の散乱の影響ともに無視されている。
しかしながら、間隙水による散乱の影響は電荷量から考えて粘土分子による散乱よりも影響が小さいと考えられる。
また、粘土分子の内部構造に起因した散乱パターンは、観測方向$2\theta$が比較的大きな範囲に現れるため、
ここで興味の対象となる、粘土分子の積層構造に起因した回折ピークの位置や強度には影響がない。
以上のことからから粘土分子の内部構造や間隙水分布を無視した電荷分布を用いても、
粘土の積層構造や膨潤状態を調べるためXRDパターンの合成において支障はないと言える。
なお、メソMD結果からXRDパターンを合成する際には、特性関数$\Gamma(\fat{x})$を適当な空間解像度で
ピクセル画像として構成する。その空間変数に関する2次元フーリエ変換を離散フーリエ変換によって
計算すれば、全ての波数ベクトル$\fat{k}$に対して式(\ref{eqn:Iphi})右辺を得ることができる。
波数ベクトル$\fat{k}$の方向$2\theta$に応じて計算されたフーリエスペクトルを
サンプリングすれば、各$\theta$に対する$I_\varphi(2\theta)$を得ることができる。
波数空間におけるこのようなサンプリングを入射方向$\varphi$を変えて行い、その結果を
加算すれば、XRD試験で観測される回折パターン$I(2\theta)$を合成することができる。
%%%%%%%%%
% XRD pattern example, Binary pixel image & 2D FFT
%%%%%%%%
\subsection{動径分布関数}
動径分布関数(radial distribution function: RDF)は、
着目粒子から一定の距離だけ離れた位置に存在する粒子数を表すものである。
正確には、ある粒子を原点として測った動径距離$r$について,
区間$[r,\Delta r)$に位置する粒子の数を$g(r,r+\Delta r)$とするとき、
2次元空間における動径分布関数$f(r)$は
\begin{equation}
	f(r):=\frac{g(r,r+\Delta r)}{2\pi r}
	\label{eqn:RDF}
\end{equation}
で与えられる.
ここで、第$i$番目の粗視化粒子の位置を$\fat{r}_i$とし、区間$[r,r+\Delta r)$で
1,そのほかの位置では0をとる矩形関数を
\begin{equation}
	U(r;\Delta r):=\left\{
		\begin{array}{cc}
			1 &  r \in [r;r+\Delta r)\\
			0 &  otherwise
		\end{array}
	\right.
	\label{eqn:}
\end{equation}
と表すと、$g(r,r+\Delta r)$は$U(r;\Delta r)$を用いて次のように書くことができる。
\begin{equation}
	g(r,\,r+\Delta r)=\sum_{i\neq j} U(r_{ij}-r;\Delta r)
	\label{eqn:gr}
\end{equation}
ただし、
\begin{equation}
	r_{ij}=\left|\fat{r}_i-\fat{r}_j\right|
	\label{eqn:rij}
\end{equation}
とする。式(\ref{eqn:RDF})で定義されるRDFを、組織構造解析にそのまま用いると、粘土分子を構成する
粗視化粒子の規則的な配列に起因した顕著なピークが現れ、粘土分子の積層構造の形成に関する情報を読みとることができない。
%%
%% RDF失敗例
%%
そこで、積層構造に寄与する粗視化分子が優先的にカウントされるように、RDFの定義を以下のように修正する。

粗視化粒子$i$から粒子$j$を望む方向を指す単位ベクトルを
\begin{equation}
	\hat{\fat{r}}_{ij}=\frac{\fat{r}_j-\fat{r}_i}{r_{ij}}
	\label{eqn:rhat}
\end{equation}
と表す。各々の粗視化粒子は、それが属する粘土分子の情報を参照することで、
粒子の向きが定められる。粒子$i$の向きをそれが属する粘土分子の法線ベクトル$\fat{n}_i$で定めると、
$\hat{\fat{r}}_{ij}\cdot\fat{n}_i$
と
$\hat{\fat{r}}_{ji}\cdot\fat{n}_j$
は、粒子$i$と$j$が近接していて同一分子に属するときには小さな値をとる傾向にある。
一方、2つの粒子が互いに積層した分子に属し、かつ近接している場合には、これらは大きな値を取る。
そこで、粒子数をカウントする際にこれらの量で重み付けが行われるように、
\begin{equation}
	f(r)=
	\sum_{i\neq j} \frac{U(r_{ij}-r;\, \Delta r)}{2\pi r}
	\left\{
		(\hat{\fat{r}}_{ij} \cdot \fat{n}_i)
		(\hat{\fat{r}}_{ji} \cdot \fat{n}_j)
	\right\}^2
	\label{eqn:RDFn}
\end{equation}
と動径分布関数を定義すれば、積層した分子に属する粗視化粒子が、動径距離に応じてどのように分布しているかを
調べる目的に適したRDFを構成することができる。
%%
%% RDF 成功例
%%
%粘土分子の積層数を評価するための動径分布関数を定義して数値的に求め、積層数の見積りを得る。
\subsection{局所数密度分布}
質点系を構成する第$i$番目の粒子の質量を$m_i$,位置を$\fat{r}_i$とすれば、
質量密度分布$\rho(\fat{x})$は、ディラクのデルタ関数を用いて
\begin{equation}
	\rho(\fat{x})=\sum_{i} m_i\delta\left(\fat{x}-\fat{r}_i\right)
\end{equation}
と表される。これを一般化した
\begin{equation}
	\rho(\fat{x})=\sum_{i} m_i w\left(\fat{x}-\fat{r}_i\right)
\end{equation}
を、ここでの組織構造解析の局所密度として用いる。ただし、$w(\fat{x})$は
\begin{equation}
	\int w(\fat{x}) d\fat{x} =1
\end{equation}
と規格化された、正または零の値を取る重み関数を表す。
ここでは、$w(\fat{x})$として、平均が0, 標準偏差が$\sigma$の2次元正規分布
\begin{equation}
	N(\fat{x};\sigma):=
	\frac{1}{2\pi\sigma^2}
	\exp\left(
		\frac{\left|\fat{x}\right|^2}{2\sigma^2}
	\right)
	\label{eqn:Gss2D}
\end{equation}
を用いる。これにより、平滑化された密度場が得られ、式(\ref{eqn:Gss2D})において
$\sigma$を層間距離程度にとれば、積層した粘土分子層間は、層外の空隙よりも
高い密度を持つとして評価されるような局所密度を与えることができる。
粘土分子層間と層外の間隙を区別するために、一般化された局所密度を定義する。
%粘土分子が占める領域を固相領域$D_s$、それ以外の間隙部を$D_p$とする。
%$D_s$における質量密度は粘土の密度$\rho_s$に一致し、$D_p$における密度は0とみなすことができる。
%すなわち、メソMD解析結果から与えられる局所密度$\rho(\fat{x})$は、特性関数$\Gamma(\fat{x})$
%を用いて
%\begin{equation}
%	\rho(\fat{x})=\rho_s \Gamma(\fat{x})
%\end{equation}
%と表すことができる。
\subsection{粘土分子方向の確率密度}
組織構造の配向性は、拡散や透水などの物質輸送挙動の異方性の原因になると予想される。
そこで、メソMD解析の結果として得られた組織構造がもつ配向性を定量的に調べることを考える。
各粗視化粒子は、それが属する分子の情報を参照することで、粒子の向きを定めることができることは、上に述べたとおりである。そこで、粒子$i$の向きを$\fat{n}_i$と単位ベクトルで表し、単位ベクトルの
偏角を$\alpha_i = \alpha_i( \fat{n}_i)$と書けば、$\left\{ \alpha_i \right\}$のヒストグラムを正規化して
粒子方向$\alpha$出現頻度を表す確率密度関数
\begin{equation}
	\Psi:=
	{\rm Prob}
	\left(\alpha \left| \left\{\alpha_i \right\} \right. \right)
	\label{eqn:PDF}
\end{equation}
を得ることができる。このような確率密度の分布状況をみることで、系の配向性について定量的な評価を与えることができる。
また、確率密度を推定するためのサンプルを、特定の領域$\cal R$にある粒子に限定することで、局所的な配向性を
調べることも可能である。その場合は、
\begin{equation}
	\Psi({\cal R}):=
	{\rm Prob} \left(\alpha \left| \left\{\alpha_i \right\}_{\fat{r}_i\in {\cal R}} \right. \right)
	\label{eqn:PDF_R}
\end{equation}
%粘土分子が積層構造を作ることにより、粘土含水系が全体としてどのような配向性を持つかを調べる方法を与える。
